%%%%%%%%%%%%%%%%%%%%%%%%%%%%%%%%%%%%%%%%%%%%%%%%%%%%%%%%%%%%%%%%%%%%%%%%%%%%%%%%%%%%%%%%%
%             template.tex                                                              %
%                                                                                       %
%            Author: Sergej Lewin 10/2008                                               %
%                                                                                       %    
% !!!Man braucht noch die Datei Ueb.sty (im gleichen Ordner wie die Hauptdatei)!!!      %
%%%%%%%%%%%%%%%%%%%%%%%%%%%%%%%%%%%%%%%%%%%%%%%%%%%%%%%%%%%%%%%%%%%%%%%%%%%%%%%%%%%%%%%%%
\documentclass[a4paper,11pt]{article}             % bestimmt das Aussehen eines Dokuments
\usepackage{Ueb}                                  % vordefinierte Makros

%!!!!anpassen an das Betriebssystem!!!, um Umlaute zu verwenden
\usepackage[utf8]{inputenc}                      %Linux
%\usepackage[latin1]{inputenc}                    %Windows
%\usepackage[applemac]{inputenc}                  %Mac



%Namen und Matrikelnummern anpassen
\zweinamen{Aristide Voufouo}{Alexander Schlüter} %2er Gruppen
% \dreinamen{Tanja Wilke}{Jan Rathner}{Alexander Schlüter} %3er Gruppe

%Briefkastennummer anpassen. z. B. \briefkasten{104}
% \briefkasten{83}

%Termin der Uebungsgruppe und Raum anpassen z. B. \termin{Mo. 12-14 , SR2}
\termin{Fr. 10-12, SRZ205}

%Blattnummer anpassen z. B. \blatt{5}
\blatt{2}

\begin{document}
%Hier kommt der Text des Dokuments......
\setcounter{excnt}{5}
\begin{ex}
\begin{exlist}
  \leavevmode
  \setcounter{exlisti}{2}
  \item
Aufgrund klar erkennbarer Muster sind die Zufallsgeneratoren für
\begin{equation*}
a\in\Set{6,16,22,33,35,36,47,50,61,62,64,75,81,91,96}
\end{equation*}
 auszuschließen.
\end{exlist}
\end{ex}
\begin{ex}
\begin{exlist}
  \setcounter{exlisti}{2}
  \leavevmode
  \item 
    \begin{exlist}
\item Nach dem Chinesischen Restsatz können alle ganze Zahlen in 
\begin{equation*}
\ZZ\cap [0,
  m_1\cdot m_2\cdot m_3)=\ZZ\cap[0, 999999000)
  \end{equation*}
  eindeutig durch ihre Reste bezüglich $m_1$, $m_2$ und $m_3$ dargestellt
  werden, denn die Moduli sind teilerfremd:
\begin{align*}
999 &= 3^3\cdot 37 \\
1000 &= 2^3\cdot 5^3 \\
1001 &= 7 \cdot 11\cdot 13 
\end{align*}
\item Formal werden die Reste bezüglich der Moduli ausgerechnet:
\begin{align*}
  r_1&=r\bmod 999 \\
r_2&=r\bmod1000 \\
r_3 &= r\bmod1001
  \end{align*}
  Man kann auch nach folgendem Schema rechnen:
  \begin{table}[h!]
    \centering
\begin{tabular}{l l}
  $r\bmod 999$: & Addiere Tripel von Ziffern und nehme die sich ergebende Summe $\mod 999$ \\
$r\bmod 1000$ & Nehme die drei am weitesten rechts stehenden Ziffern \\
$r\bmod 1001$ & Alternierend addiere und subtrahiere Tripel von Ziffern und \\ &nehme das Resultat $\mod 1001$
  \end{tabular}
  \end{table}
  \item Es gilt
    \begin{equation*}
x_{44} = 701408733<999999000=m_1m_2m_3<1134903170=x_{45}
\end{equation*}
also ist die 44. Fibonacci-Zahl die größte, die dargestellt werden kann.
\end{exlist}
\end{exlist}
\end{ex}
\end{document}
