%%%%%%%%%%%%%%%%%%%%%%%%%%%%%%%%%%%%%%%%%%%%%%%%%%%%%%%%%%%%%%%%%%%%%%%%%%%%%%%%%%%%%%%%%
%             template.tex                                                              %
%                                                                                       %
%            Author: Sergej Lewin 10/2008                                               %
%                                                                                       %    
% !!!Man braucht noch die Datei Ueb.sty (im gleichen Ordner wie die Hauptdatei)!!!      %
%%%%%%%%%%%%%%%%%%%%%%%%%%%%%%%%%%%%%%%%%%%%%%%%%%%%%%%%%%%%%%%%%%%%%%%%%%%%%%%%%%%%%%%%%
\documentclass[a4paper,11pt]{article}             % bestimmt das Aussehen eines Dokuments
\usepackage{Ueb}                                  % vordefinierte Makros

%!!!!anpassen an das Betriebssystem!!!, um Umlaute zu verwenden
\usepackage[utf8]{inputenc}                      %Linux
%\usepackage[latin1]{inputenc}                    %Windows
%\usepackage[applemac]{inputenc}                  %Mac



%Namen und Matrikelnummern anpassen
\zweinamen{Aristide Voufouo}{Alexander Schlüter} %2er Gruppen
% \dreinamen{Tanja Wilke}{Jan Rathner}{Alexander Schlüter} %3er Gruppe

%Briefkastennummer anpassen. z. B. \briefkasten{104}
% \briefkasten{83}

%Termin der Uebungsgruppe und Raum anpassen z. B. \termin{Mo. 12-14 , SR2}
\termin{Fr. 10-12, SRZ205}

%Blattnummer anpassen z. B. \blatt{5}
\blatt{4}

\newcommand{\mr}[1]{\multirow{2}{*}{$#1$}}

\begin{document}
%Hier kommt der Text des Dokuments......
\setcounter{excnt}{13}
\begin{ex}
  \leavevmode
  \begin{exlist}
  \item \label{ex:14a}  
   \textbf{Voraussetzung:} Gegeben sei $g:\NN_0\mapsto\RR^+$. Definiere
     \begin{align*}
      \mathcal{O}_1(g) &\coloneqq \set{f:\NN_0\mapsto\RR^+ | \exists c\in\RR^+:\forall n\in\NN_0:f(n)\leq c\cdot g(n)} \\
      \mathcal{O}_2(g) &\coloneqq \set{f:\NN_0\mapsto\RR^+ | \exists c\in\RR^+:\exists n_0\in\NN_0:\forall n\geq n_0:f(n)\leq c\cdot g(n)} \\
     \end{align*}
  \textbf{Behauptung:} Es gilt $\mathcal{O}_1(g)=\mathcal{O}_2(g)$. \\
\textbf{Beweis:}
Sei $f\in\mathcal{O}_1(g)$. Dann existiert $c\in\RR^+$, sodass für alle
$n\in\NN_0$ gilt: $f(n)\leq c\cdot g(n)$. Setze $n_0\coloneqq 0$. Dann ist
offensichtlich die Bedingung für $\mathcal{O}_1(g)$ erfüllt und
$f\in\mathcal{O}_2(g)$.

Sei nun $f\in\mathcal{O}_2(g)$. Seien $c, n_0$ entsprechend der Bedingung in
$\mathcal{O}_2(g)$ gegeben. Da $g$ nur positive Werte annimmt können wir definieren
\begin{equation*}
 c'\coloneqq \max\left( \Set{c}\cup\Set{\frac{f(n)}{g(n)} | n < n_0}\right)\, .
\end{equation*}
Wegen $c'\geq c$ und $g>0$ ist die Bedingung $f(n)\leq c\cdot g(n)\leq c' g(n)$ für alle $n\geq n_0$
erfüllt. Ist $n < n_0$, so gilt nach Definition von $c'$
\begin{equation*}
  \frac{f(n)}{g(n)}\leq c'\implies f(n) \leq c'\cdot g(n)
\end{equation*}
und insgesamt folgt $f\in\mathcal{O}_1(g)$.
\item 
\textbf{Voraussetzung:} $f,g:\NN_0\mapsto\RR^+$ mit
\begin{equation*}
  \lim_{n\to\infty}\frac{f(n)}{g(n)} = 0 
\end{equation*}
\textbf{Behauptung:} $f\in\ON{g}$ \\
\textbf{Beweis:} Nach Definition des Limes existiert für $\epsilon =1$ ein
$n_0\in\NN_0$, sodass für alle $n\geq n_0$ gilt:
\begin{equation*}
 \abs*{\frac{f(n)}{g(n)}} <\epsilon = 1\quad \text{d.h.}\quad f(n) < g(n) 
\end{equation*}
Also ist die Bedingung der zweiten Definition aus \ref{ex:14a} erfüllt und
$f\in\ON{g}$. 
\item\leavevmode 
  % \begin{center}
\begin{tabular}{*{9}{c|}c}
\hline
\mr{\left(\sfrac{1}{3}\right)^n} & \mr{6} & $\log_2 n$ & \mr{\log^2 n} & \mr{n^{1/3}+\log_2 n} & \mr{\sqrt{n}} & \mr{n / \log_2 n} & \mr{n} & \mr{n\log_2 n} & $n^2$ \\
& & $\ln n$ &  & & & & & & $n^2+\log_2 n$ \\
  % \hhline{*{4}{=:}=:t:t:t:}
  \hline
\end{tabular}
\newline
\vspace{5mm}
\newline
\begin{tabular}{*{4}{c|}c}
\hline
  $n^3$ & $n - n^3 + 7n^5$ & $\left(\sfrac{3}{2}\right)^n$ & $2^n$ & $n!$ \\
  \hline
\end{tabular}

% \end{center}
  \end{exlist}
\end{ex}
\end{document}
