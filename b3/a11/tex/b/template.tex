%%%%%%%%%%%%%%%%%%%%%%%%%%%%%%%%%%%%%%%%%%%%%%%%%%%%%%%%%%%%%%%%%%%%%%%%%%%%%%%%%%%%%%%%%
%             template.tex                                                              %
%                                                                                       %
%            Author: Sergej Lewin 10/2008                                               %
%                                                                                       %    
% !!!Man braucht noch die Datei Ueb.sty (im gleichen Ordner wie die Hauptdatei)!!!      %
%%%%%%%%%%%%%%%%%%%%%%%%%%%%%%%%%%%%%%%%%%%%%%%%%%%%%%%%%%%%%%%%%%%%%%%%%%%%%%%%%%%%%%%%%
\documentclass[a4paper,11pt]{article}             % bestimmt das Aussehen eines Dokuments
\usepackage{Ueb}                                  % vordefinierte Makros

%!!!!anpassen an das Betriebssystem!!!, um Umlaute zu verwenden
\usepackage[utf8]{inputenc}                      %Linux
%\usepackage[latin1]{inputenc}                    %Windows
%\usepackage[applemac]{inputenc}                  %Mac



%Namen und Matrikelnummern anpassen
\zweinamen{Aristide Voufouo}{Alexander Schlüter} %2er Gruppen
% \dreinamen{Tanja Wilke}{Jan Rathner}{Alexander Schlüter} %3er Gruppe

%Briefkastennummer anpassen. z. B. \briefkasten{104}
% \briefkasten{83}

%Termin der Uebungsgruppe und Raum anpassen z. B. \termin{Mo. 12-14 , SR2}
\termin{Fr. 10-12, SRZ205}

%Blattnummer anpassen z. B. \blatt{5}
\blatt{3}

\begin{document}
%Hier kommt der Text des Dokuments......
\setcounter{excnt}{10}
\begin{ex}
\begin{exlist}
  \leavevmode
  \setcounter{exlisti}{1}
\item 
  Für die Implementation in Java bietet es sich an, das oben angegebene
  Verfahren mit der Binärdarstellung statt der Dezimaldarstellung der Integers
  durchzuführen. Seien $s_0, s_1, s_2$ die Vorzeichenbits von $i_0, i_1, i_2$,
  so baue den Integer $x$ so zusammen:
  \begin{equation*}
 x = \dots p_2r_1q_1p_1r_{0}q_{0}p_{0}s_2s_1s_0
  \end{equation*}
  Diesmal stehen die Buchstaben $p, q, r$ für die Bitfolgen der Ausgangswerte.

Da das Ergebnis $x$ wieder in einem Integer mit 32 Bits gespeichert werden soll
und 3 Bits für die Vorzeichen gebraucht werden, bleiben 29 Bits übrig.
Das bedeutet bei gleichberechtigter Auteilung, dass von $i_0$ und $i_1$ je 10
Bits gespeichert werden können, von $i_2$ nur 9.
Da in Java negative Integers nach dem Zweierkomplement kodiert werden, verbrauchen
betraglich kleine negative Integers viele Bitstellen, z.B. wird $-1$
repräsentiert als einen Bitstring mit 32 Einsen. Bei der Speicherung von $x$ in
einem Integer werden die Bitfolgen auf Länge 9 bzw. 10 gekürzt. Um betraglich
kleine Werte korrekt zu dekodieren, müssen folglich negative Werte links mit
Einsen aufgefüllt werden, positive mit Nullen. Der Code ist an der
entsprechenden Stelle kommentiert.
\item + (d) Nichts fällt auf, beide Tests sind erfolgreich :-) 
\end{exlist}
\end{ex}
\end{document}
