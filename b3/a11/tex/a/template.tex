%%%%%%%%%%%%%%%%%%%%%%%%%%%%%%%%%%%%%%%%%%%%%%%%%%%%%%%%%%%%%%%%%%%%%%%%%%%%%%%%%%%%%%%%%
%             template.tex                                                              %
%                                                                                       %
%            Author: Sergej Lewin 10/2008                                               %
%                                                                                       %    
% !!!Man braucht noch die Datei Ueb.sty (im gleichen Ordner wie die Hauptdatei)!!!      %
%%%%%%%%%%%%%%%%%%%%%%%%%%%%%%%%%%%%%%%%%%%%%%%%%%%%%%%%%%%%%%%%%%%%%%%%%%%%%%%%%%%%%%%%%
\documentclass[a4paper,11pt]{article}             % bestimmt das Aussehen eines Dokuments
\usepackage{Ueb}                                  % vordefinierte Makros

%!!!!anpassen an das Betriebssystem!!!, um Umlaute zu verwenden
\usepackage[utf8]{inputenc}                      %Linux
%\usepackage[latin1]{inputenc}                    %Windows
%\usepackage[applemac]{inputenc}                  %Mac



%Namen und Matrikelnummern anpassen
\zweinamen{Aristide Voufouo}{Alexander Schlüter} %2er Gruppen
% \dreinamen{Tanja Wilke}{Jan Rathner}{Alexander Schlüter} %3er Gruppe

%Briefkastennummer anpassen. z. B. \briefkasten{104}
% \briefkasten{83}

%Termin der Uebungsgruppe und Raum anpassen z. B. \termin{Mo. 12-14 , SR2}
\termin{Fr. 10-12, SRZ205}

%Blattnummer anpassen z. B. \blatt{5}
\blatt{3}

\begin{document}
%Hier kommt der Text des Dokuments......
\setcounter{excnt}{10}
\begin{ex}
\begin{exlist}
  \leavevmode
\item
Es seien drei Integer-Werte in Dezimaldarstellung gegeben:
\begin{align*}
 i_0&=\epsilon_0\cdot p_{l}\dots p_{2}p_{1}p_{0} \\
 i_1&=\epsilon_1\cdot q_{m}\dots q_{2}q_{1}q_{0} \\
 i_2&=\epsilon_2\cdot r_{n}\dots r_{2}r_{1}r_{0}
\end{align*}
Hierbei seien $\epsilon_0,\epsilon_1,\epsilon_2\in\set{-1, 1}$ die Vorzeichen,
$l, n, m\in\NN$ die Ziffernlängen und ganz rechts stehen die am wenigsten
signifikanten Ziffern.
Da es $2^3=8$ mögliche Kombinationen von Vorzeichen gibt, können die
Vorzeichen in der ersten Dezimalziffer $d_0\in\set{0, 1,\cdots, 7}$ gespeichert
werden. Zum Beispiel kann der Bitstring aus den Vorzeichenbits $s_2s_1s_0$ als
Binärzahl interpretiert werden und die entsprechende Dezimalzahl gespeichert werden.

Baue nun einen Integer-Wert $x$ zusammen, dessen erste (rechte) Ziffer $d_0$
ist und der danach abwechselnd Ziffern von $i, j$ und $k$ enthält, also
\begin{equation*}
 x = \dots p_2r_1q_1p_1r_{0}q_{0}p_{0}d_0 
\end{equation*}
Ist die Ziffernlänge von einem der Ausgangswerte länger als die eines anderen, so
muss der kürzere mit Nullen aufgefüllt werden. Zum Beispiel werden $i=12,
\,j=345,\, k=678$ codiert als
\begin{equation*}
 x = 8507426310 \qquad ,
\end{equation*}
wobei die ganz rechte Null sich aus den Vorzeichen ergibt.
\end{exlist}
\end{ex}
\end{document}
