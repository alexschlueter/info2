%%%%%%%%%%%%%%%%%%%%%%%%%%%%%%%%%%%%%%%%%%%%%%%%%%%%%%%%%%%%%%%%%%%%%%%%%%%%%%%%%%%%%%%%%
%             template.tex                                                              %
%                                                                                       %
%            Author: Sergej Lewin 10/2008                                               %
%                                                                                       %    
% !!!Man braucht noch die Datei Ueb.sty (im gleichen Ordner wie die Hauptdatei)!!!      %
%%%%%%%%%%%%%%%%%%%%%%%%%%%%%%%%%%%%%%%%%%%%%%%%%%%%%%%%%%%%%%%%%%%%%%%%%%%%%%%%%%%%%%%%%
\documentclass[a4paper,11pt]{article}             % bestimmt das Aussehen eines Dokuments
\usepackage{Ueb}                                  % vordefinierte Makros

%!!!!anpassen an das Betriebssystem!!!, um Umlaute zu verwenden
\usepackage[utf8]{inputenc}                      %Linux
%\usepackage[latin1]{inputenc}                    %Windows
%\usepackage[applemac]{inputenc}                  %Mac



%Namen und Matrikelnummern anpassen
\zweinamen{Aristide Voufouo}{Alexander Schlüter} %2er Gruppen
% \dreinamen{Tanja Wilke}{Jan Rathner}{Alexander Schlüter} %3er Gruppe

%Briefkastennummer anpassen. z. B. \briefkasten{104}
% \briefkasten{83}

%Termin der Uebungsgruppe und Raum anpassen z. B. \termin{Mo. 12-14 , SR2}
\termin{Fr. 10-12, SRZ205}

%Blattnummer anpassen z. B. \blatt{5}
\blatt{3}

\begin{document}
%Hier kommt der Text des Dokuments......
\setcounter{excnt}{7}
\begin{ex}
\begin{exlist}
  \leavevmode
  \item Die Adjazenzmatrix des Graphen $\mathcal{G}$ ist
    \begin{equation*}
A= \begin{bmatrix}                
      \text{F} & \text{F} & \text{T} &  \text{F} & \text{F} & \text{F} \\
      \text{F} & \text{F} & \text{T} &  \text{F} & \text{T} &  \text{F} \\
      \text{F} & \text{F} & \text{F} & \text{T} &  \text{F} & \text{T} \\
      \text{F} & \text{F} & \text{T} &  \text{F} & \text{F} & \text{F} \\
      \text{F} & \text{T} &  \text{F} & \text{F} & \text{F} & \text{F} \\
      \text{T} &  \text{F} & \text{F} & \text{F} & \text{T} &  \text{F}
\end{bmatrix}
    \end{equation*}
  \item Die Zwischenergebnisse aus dem Algorithmus von Warshall sind
    \begin{align*}
B^{-1} &=
\begin{bmatrix}
\text{T} & \text{F} & \text{T} & \text{F} & \text{F} & \text{F}  \\
\text{F} & \text{T} & \text{T} & \text{F} & \text{T} & \text{F}  \\
\text{F} & \text{F} & \text{T} & \text{T} & \text{F} & \text{T}  \\
\text{F} & \text{F} & \text{T} & \text{T} & \text{F} & \text{F}  \\
\text{F} & \text{T} & \text{F} & \text{F} & \text{T} & \text{F}  \\
\text{T} & \text{F} & \text{F} & \text{F} & \text{T} & \text{T} 
\end{bmatrix}
& B^{0} &=
\begin{bmatrix}
\text{T} & \text{F} & \text{T} & \text{F} & \text{F} & \text{F}  \\
\text{F} & \text{T} & \text{T} & \text{F} & \text{T} & \text{F}  \\
\text{F} & \text{F} & \text{T} & \text{T} & \text{F} & \text{T}  \\
\text{F} & \text{F} & \text{T} & \text{T} & \text{F} & \text{F}  \\
\text{F} & \text{T} & \text{F} & \text{F} & \text{T} & \text{F}  \\
\text{T} & \text{F} & \text{T} & \text{F} & \text{T} & \text{T} 
\end{bmatrix} \\
B^{1} &=
\begin{bmatrix}
\text{T} & \text{F} & \text{T} & \text{F} & \text{F} & \text{F}  \\
\text{F} & \text{T} & \text{T} & \text{F} & \text{T} & \text{F}  \\
\text{F} & \text{F} & \text{T} & \text{T} & \text{F} & \text{T}  \\
\text{F} & \text{F} & \text{T} & \text{T} & \text{F} & \text{F}  \\
\text{F} & \text{T} & \text{T} & \text{F} & \text{T} & \text{F}  \\
\text{T} & \text{F} & \text{T} & \text{F} & \text{T} & \text{T} 
\end{bmatrix}
& B^{2} &=
\begin{bmatrix}
\text{T} & \text{F} & \text{T} & \text{T} & \text{F} & \text{T}  \\
\text{F} & \text{T} & \text{T} & \text{T} & \text{T} & \text{T}  \\
\text{F} & \text{F} & \text{T} & \text{T} & \text{F} & \text{T}  \\
\text{F} & \text{F} & \text{T} & \text{T} & \text{F} & \text{T}  \\
\text{F} & \text{T} & \text{T} & \text{T} & \text{T} & \text{T}  \\
\text{T} & \text{F} & \text{T} & \text{T} & \text{T} & \text{T} 
\end{bmatrix} \\
B^{3} &=
\begin{bmatrix}
\text{T} & \text{F} & \text{T} & \text{T} & \text{F} & \text{T}  \\
\text{F} & \text{T} & \text{T} & \text{T} & \text{T} & \text{T}  \\
\text{F} & \text{F} & \text{T} & \text{T} & \text{F} & \text{T}  \\
\text{F} & \text{F} & \text{T} & \text{T} & \text{F} & \text{T}  \\
\text{F} & \text{T} & \text{T} & \text{T} & \text{T} & \text{T}  \\
\text{T} & \text{F} & \text{T} & \text{T} & \text{T} & \text{T} 
\end{bmatrix}
& B^{4} &=
\begin{bmatrix}
\text{T} & \text{F} & \text{T} & \text{T} & \text{F} & \text{T}  \\
\text{F} & \text{T} & \text{T} & \text{T} & \text{T} & \text{T}  \\
\text{F} & \text{F} & \text{T} & \text{T} & \text{F} & \text{T}  \\
\text{F} & \text{F} & \text{T} & \text{T} & \text{F} & \text{T}  \\
\text{F} & \text{T} & \text{T} & \text{T} & \text{T} & \text{T}  \\
\text{T} & \text{T} & \text{T} & \text{T} & \text{T} & \text{T} 
\end{bmatrix} \\
B^{5} &=
\begin{bmatrix}
\text{T} & \text{T} & \text{T} & \text{T} & \text{T} & \text{T}  \\
\text{T} & \text{T} & \text{T} & \text{T} & \text{T} & \text{T}  \\
\text{T} & \text{T} & \text{T} & \text{T} & \text{T} & \text{T}  \\
\text{T} & \text{T} & \text{T} & \text{T} & \text{T} & \text{T}  \\
\text{T} & \text{T} & \text{T} & \text{T} & \text{T} & \text{T}  \\
\text{T} & \text{T} & \text{T} & \text{T} & \text{T} & \text{T} 
\end{bmatrix}\end{align*}
\item Die Distanzmatrix von $\mathcal{G}'$ ist 
\begin{equation*}
  L=
\begin{bmatrix}
0 & 1 & 9 & \infty & \infty & \infty  \\
\infty & 0 & 6 & 4 & 1 & \infty  \\
\infty & \infty & 0 & 1 & \infty & 1  \\
\infty & \infty & 1 & 0 & \infty & \infty  \\
\infty & 1 & \infty & \infty & 0 & \infty  \\
1 & \infty & \infty & \infty & 1 & 0 
\end{bmatrix}
\end{equation*}
\item Die Zwischenergebnisse aus dem Algorithmus von Floyd sind
  \begin{align*}
D^{-1} &=
\begin{bmatrix}
0 & 1 & 9 & \infty & \infty & \infty  \\
\infty & 0 & 6 & 4 & 1 & \infty  \\
\infty & \infty & 0 & 1 & \infty & 1  \\
\infty & \infty & 1 & 0 & \infty & \infty  \\
\infty & 1 & \infty & \infty & 0 & \infty  \\
1 & \infty & \infty & \infty & 1 & 0 
\end{bmatrix}
& D^{0} &=
\begin{bmatrix}
0 & 1 & 9 & \infty & \infty & \infty  \\
\infty & 0 & 6 & 4 & 1 & \infty  \\
\infty & \infty & 0 & 1 & \infty & 1  \\
\infty & \infty & 1 & 0 & \infty & \infty  \\
\infty & 1 & \infty & \infty & 0 & \infty  \\
1 & 2 & 10 & \infty & 1 & 0 
\end{bmatrix} \\
D^{1} &=
\begin{bmatrix}
0 & 1 & 7 & 5 & 2 & \infty  \\
\infty & 0 & 6 & 4 & 1 & \infty  \\
\infty & \infty & 0 & 1 & \infty & 1  \\
\infty & \infty & 1 & 0 & \infty & \infty  \\
\infty & 1 & 7 & 5 & 0 & \infty  \\
1 & 2 & 8 & 6 & 1 & 0 
\end{bmatrix}
& D^{2} &=
\begin{bmatrix}
0 & 1 & 7 & 5 & 2 & 8  \\
\infty & 0 & 6 & 4 & 1 & 7  \\
\infty & \infty & 0 & 1 & \infty & 1  \\
\infty & \infty & 1 & 0 & \infty & 2  \\
\infty & 1 & 7 & 5 & 0 & 8  \\
1 & 2 & 8 & 6 & 1 & 0 
\end{bmatrix} \\
D^{3} &=
\begin{bmatrix}
0 & 1 & 6 & 5 & 2 & 7  \\
\infty & 0 & 5 & 4 & 1 & 6  \\
\infty & \infty & 0 & 1 & \infty & 1  \\
\infty & \infty & 1 & 0 & \infty & 2  \\
\infty & 1 & 6 & 5 & 0 & 7  \\
1 & 2 & 7 & 6 & 1 & 0 
\end{bmatrix}
& D^{4} &=
\begin{bmatrix}
0 & 1 & 6 & 5 & 2 & 7  \\
\infty & 0 & 5 & 4 & 1 & 6  \\
\infty & \infty & 0 & 1 & \infty & 1  \\
\infty & \infty & 1 & 0 & \infty & 2  \\
\infty & 1 & 6 & 5 & 0 & 7  \\
1 & 2 & 7 & 6 & 1 & 0 
\end{bmatrix} \\
D^{5} &=
\begin{bmatrix}
0 & 1 & 6 & 5 & 2 & 7  \\
7 & 0 & 5 & 4 & 1 & 6  \\
2 & 3 & 0 & 1 & 2 & 1  \\
3 & 4 & 1 & 0 & 3 & 2  \\
8 & 1 & 6 & 5 & 0 & 7  \\
1 & 2 & 7 & 6 & 1 & 0 
\end{bmatrix}\end{align*}
\end{exlist}
\end{ex}
\end{document}
